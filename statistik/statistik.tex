\documentclass[11pt,a4paper, english, swedish
]{article}
\pdfoutput=1

\usepackage{custom_as}

\graphicspath{ {figurer/} }

%%Drar in tabell och figurtexter
\usepackage[margin=10 pt]{caption}
%%För att lägga in 'att göra'-noteringar i texten
\usepackage{todonotes} %\todo{...}

%%För att själv bestämma marginalerna. 
\usepackage[
%            top    = 3cm,
%            bottom = 3cm,
%            left   = 3cm, right  = 3cm
]{geometry}

%%För att ändra hur rubrikerna ska formateras
%\renewcommand{\thesection}{...}




\begin{document}



%%%%%%%%%%%%%%%%% vvv Inbyggd titelsida vvv %%%%%%%%%%%%%%%%%

\title{ Behandling av experimentella resultat\\ 
\Large En introduktion för fysikstudenter till hur man ska hantera
mätresultat och feluppskattningar}
\author{Andréas Sundström}
\date{\today}

\maketitle

%%%%%%%%%%%%%%%%% ^^^ Inbyggd titelsida ^^^ %%%%%%%%%%%%%%%%%

\subsection*{Förord}
\small
Det här kompendiet är skrivet som en introduktion till
feluppskattningar för det svenska IPhO-laget. Jag har försökt fatta
mig kort, men jag har en tendens att kunna dra iväg när jag
skriver. 

Det viktigaste innehållet finns i avsnitt~\ref{sec:feluppskattningar},
men jag tyckte att det behövs en liten teoretisk bakgrund också, den
finns i avsnitt~\ref{sec:statistik}. Det avsnittet ska man läsa om man
vill få en lite klarare bild om varför feluppskattningarna på det ena
eller andra viset. Sedan finns har jag även lagt till ett avsnitt om
approximationer. Detta avsnitt hör inte så mycket ihop med resten av
det här kompendiet, men det är matnyttigt att kunna göra
uppskattningar både för teoretiska och experimentella beräkningar. 
\begin{flushright}
Andréas Sunsdtröm\\ Göteborg, 2016-06-10
\end{flushright}
\normalsize

\addtocounter{section}{-1}
\section{Approximationer}


\section{Statistik}\label{sec:statistik}
När man gör mätningar vill man oftast ta reda på någon storhets värde
-- exempelvis periodtiden på en pendel. Med en mätning kan man dock
inte säga något om hur bra mätningen var. Men med flera mätningar kan
vi börja göra statistik över dem. Då 

Jag vill även betona vikten av statistik i framtida vetenskapliga och
tekniska karriärer. \emph{Statistik är ett mycket kraftfullt verktyg.}
Statistiken, när man börjar behärska den, är mycket mer än bara
''medelvärden och standardavvikelser''; den spelar en väsentlig roll i
mer eller mindre all experimentell vetenskap -- särskilt vid
feluppskattningar. 

\subsection{Sannolikhetsfördelningar och täthetsfunktioner}
Den statistik och sannolikhetslära man får lära sig i gymnasiet brukar
fördet mesta vara diskret (och ändlig). Alltså att det finns ett
ändligt antal möjliga utfall och varje utfall är distikt. Exempel på
detta är tärningskast, lottdragning och kortlekar. 

Verkligheten är dock oftast mer komplicerad än så. När man gör
mätningar handrar det oftas om mätnigar av en reellvärd
storhet som t.ex. en längd\footnotemark{}. Detta betyder att längden
kommer att anta ett reellt antal milimetrar och inte vara begränsad
till ett ändligt antal möjliga värden. För att analysera detta behövs
statistik för kontinuerliga fördelningar. 
\footnotetext{''Aha!'' tänker ni: atomer och kvantfysik gör att det
  bara finns diskreta längder. ''Nähä!'' säger jag: som fysiker måste
  man kunna hantera approximationer och veta begränsningarna i sina
  mätningar. Med en linjal eller skujtmått finns det ingen chans i
  världen att man skulle kunna stöta på problem orsakade av
  kvantfysik. Man kan alltså betrakta det som om de möjliga värden
  som längden kan anta ligger kontinuerligt.}

Sannolikhetsfördelningar karakteriseras av sin täthetsfunktion, som
brukar betecknas $f$. Täthetsfunktionen talar om hur sannolikt det är
att få ett värde i ett visst intervall. Notera dock att den
\emph{inte} kan ge sannolikheten att få ett specifikt värde eftersom
det finns (ouppräknerligt) oändligt många möjliga utfall, så
sannolikheten att få ett visst exat värde är~0.



En av de vanligaste fördelningarna är \emph{normalfördelningen}. Den
har en täthesfunktion
\begin{equation}
f(x) = \frac{1}{\sqrt{2\pi\sigma^2}} \, \ee^{-\frac{(x-\mu)^2}{2\sigma^2}},
\end{equation}
där $\mu$ är fördelningens \emph{väntevärde} och $\sigma$ är dess
\emph{standardavvikelse}. 



\section{Feluppskattningar}\label{sec:feluppskattningar}
När man vill uppskatta hur stora experimentella fel/osäkerheter man
har finns det två typer av fel och två typer av feluppskattningar. De
två typerna av fel som finns är systematiska och statistiska. Sedan är
de två typerna av feluppskattningar som man behöver kunna dels direkt
feluppskattning (av statistiska fel), dels propagering av osäkerhet. 


\subsection{Statistiska osäkerheter och hur man uppskattar dem}
Säg att det finns en storhet $x$ som vi vill mäta, men att vi i varje
mätning gör ett mätfel~$\delta{x}$. Det vi mäter blir då
\begin{equation}
\hat{x}=x+\delta{x}.
\end{equation}
Utan någon mer informationom $\delta{x}$ går det inte att säga så mycket mer
från mätningen.

\subsection{Systematiska fel}

\subsection{Felpropagering}



%\newpage
%\bibliographystyle{ieeetr}
%\bibliography{referenser}%kräver en fil som heter 'referenser.bib'          


\end{document}


