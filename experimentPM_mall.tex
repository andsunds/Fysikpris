%Mall skapad av Andréas Sundström, Göteborg, 2015-02-28.
%Neders i filen finns lite påminnelser om hur man använder vissa
%kommandon. 

%Grunden är baserad på Christian von Schults mall.
\documentclass[12pt,a4paper]{article}
\pdfoutput=1
%%%%%%%%%%%%%%%%%%%%%%%vons grund%%%%%%%%%%%%%%%%%%%%%%%
\usepackage[utf8]{inputenc}
\usepackage[T1]{fontenc}
\usepackage[swedish]{babel} %OBS! Se till att vi får rätt språk.
\usepackage{amsmath}
\usepackage{lmodern}
\usepackage{units}
\usepackage{icomma}
\usepackage{color}
\usepackage{graphicx}
\usepackage{bbm}
\newcommand{\N}{\ensuremath{\mathbbm{N}}}
\newcommand{\Z}{\ensuremath{\mathbbm{Z}}}
\newcommand{\Q}{\ensuremath{\mathbbm{Q}}}
\newcommand{\R}{\ensuremath{\mathbbm{R}}}
\newcommand{\C}{\ensuremath{\mathbbm{C}}}
\newcommand{\rd}{\ensuremath{\mathrm{d}}}
\newcommand{\id}{\ensuremath{\,\rd}}
\usepackage{hyperref}

%%%%%%%%%%%%%%%%%%%%%%%Egna tillägg%%%%%%%%%%%%%%%%%%%%%%%

%%Partiell derivata
\newcommand{\pd}{\ensuremath{\partial}}
%%Följer ISO-8601 oberoende av språk.
\usepackage{datetime} 
\newdateformat{specialdate}{\THEYEAR-\twodigit{\THEMONTH}-\twodigit{\THEDAY}}
%%Göra grader Celcius
\newcommand{\degC}{\ensuremath{\,^\circ\mathrm{C}}}
%%Figurreferenser
\newcommand{\Figref}{\figurename~\ref} %Stor bokstav i början
\newcommand{\figref}{\MakeLowercase{\figurename}~\ref} 
%%Tabellreferenser
\newcommand{\Tabref}{\tablename~\ref} %Stor bokstav i början
\newcommand{\tabref}{\MakeLowercase{\tablename}~\ref}
%%Ohm enhetskommando
\newcommand{\ohm}{\ifmmode \Omega \else $\Omega$ \fi}
%%Varepsilon är det enda rätta epsilon
\renewcommand{\epsilon}{\varepsilon}


%%%%%%%%%%%%%%%%%%Övriga matnyttiga paket%%%%%%%%%%%%%%%%%

%%För att kunna inkludera andra PDF-dokument
\usepackage{pdfpages}
%%För att kunna ha roterade bilder
\usepackage{rotating}
%%För att inkludera MATLABkod. 
%\usepackage[framed,numbered,autolinebreaks,useliterate]{mcode}
%\usepackage{listings} 
%\lstloadlanguages{matlab} 
%\lstset{language=matlab} 
%\lstset{literate= {å}{{\r{a}}}1 {ä}{{\"a}}1 {ö}{{\"o}}1 {Å}{{\r{A}}}1
%  {Ä}{{\"A}}1 {Ö}{{\"O}}1}%För att få svenska bokstäver från MATLAB.


%%För att själv bestämma marginalerna. 
%\usepackage[
%            top    = 3cm,
%            bottom = 3cm,
%            left   = 3cm, right  = 3cm
%]{geometry}




%Fixar typstorleken i \section så att den inte blir för stor
\usepackage{sectsty}
\sectionfont{\fontsize{13}{0}\selectfont}

\begin{document}
\title{}
\author{}
\date{2015-03-20}%Datum för experimentfinal 2015
\maketitle

%Jag tänker mig att numrering är onödig
\section*{Bakgrund} 
%Det kan vara bra med en kort bakrund (typ 50-75 ord).
%Ska det vara några bilder här? Det känns mest störiga.

\section*{Uppgift}
%Ha med relevanta bilder

%Tilltal i singular: ''Din uppgift är att ...''


\section*{Materiel}
%Lista all utrustning. Vid behöv förklara den.
%Här måste vi ha med bilder på all utrustning!

%En punktlista med all materiel.
%Listan bör innehålla: <antal>, <sak>, <ev. förklaring hur den
%används>

\begin{itemize}
\item 
\end{itemize}

%En längre förklaring om hur något används kan kanske läggas här,
%efter listan. 

\end{document}
%%%%%%%%Påminnelser om hur man använder vissa kommandon%%%%%%%%%%

%% På svenska ska citattecknet vara samma i både början och slut.
%% Använd två apostrofer (två enkelfjongar): ''.

%%För att referera till till tidigare fotnot:
%\footnotemark[\value{footnote}]

%% Inkludera PDF-dokument
%\includepdf[pages={1-}]{filnamn.pdf} %Filnamnet får INTE innehålla 'mellanslag'!

%% Figurer inkluderade som pdf-filer
%\begin{figure}\centering
%\centerline{ %centrerar även större bilder
%\includegraphics[width=1\textwidth]{filnamn.pdf}
%}
%\caption{\label{figuren} Perioden $T$ som funktion av pendellängden.}
%\end{figure}

%% Figurer inkluderade med xfigs "Combined PDF/LaTeX"
%\begin{figure}\centering
%\input{filnamn.pdf_t}
%\caption{\label{finafiguren} Perioden $T$ som funktion av
%  pendellängden.}
%\end{figure}


%% Figurer roterade 90 grader
%\begin{sidewaysfigure}\centering
%\centerline{ %centrerar även större bilder
%\includegraphics[width=1\textwidth]{filnamn.pdf}
%}
%\caption{\label{figuren} Perioden $T$ som funktion av pendellängden.}
%\end{sidewaysfigure}
