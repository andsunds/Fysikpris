\documentclass[11pt,a4paper, 
english, swedish %% Make sure to put the main language last!
]{article}
\pdfoutput=1

%% Andréas's custom package 
%% (Will work for most purposes, but is mainly focused on physics.)
\usepackage{custom_as}

%% Figures can now be put in a folder: 
\graphicspath{ {figurer/} %{some_folder_name/}
}

%% If you want to change the margins for just the captions
\usepackage[margin=10 pt]{caption}

%% To add todo-notes in the pdf
\usepackage[%disable  %%this will hide all notes
]{todonotes} 

%% Cange the margin in the documents
\usepackage[
%            top    = 3cm,              %% top margin
%            bottom = 3cm,              %% bottom margin
            left   = 2.5cm, right  = 2.5cm %% left and right margins
]{geometry}


%% If you want to chage the formating of the section headers
%\renewcommand{\thesection}{...}


%\setlength{\parindent}{0pt} % Default is 15pt.
%\setlength{\parskip}{9pt plus3pt minus2pt}



%%%%%%%%%%%%%%%%%%%%%%%%%%%%%%%%%%%%%%%%%%%%%%%%%%%%%%%%%%%%%%%%%%%%%%
\begin{document}%% v v v v v v v v v v v v v v v v v v v v v v v v v v
%%%%%%%%%%%%%%%%%%%%%%%%%%%%%%%%%%%%%%%%%%%%%%%%%%%%%%%%%%%%%%%%%%%%%%


%% If you want to use an external file for the title page
%\input{titlepages.tex}


%%%%%%%%%%%%%%%%%%%% vvv Internal title page vvv %%%%%%%%%%%%%%%%%%%%%
\title{Lägerrapport \\
{\Large Experimenttränigslägret i Lund, juni 2017} }
\author{Andréas Sundström}
\date{\today}

\maketitle

%%%%%%%%%%%%%%%%%%%% ^^^ Internal title page ^^^ %%%%%%%%%%%%%%%%%%%%%
%% If you want a list of all todos
%\todolist

\section*{Utrustning}
På fredagen, 9 juni, hämtade jag och Erik ut experimentutrustningarna
från Bosses föråd och bar ner dem till de bokade experimentlokalerna i
Fysicum, på Lunds universitet. 
Vi hämtade:
\begin{itemize}
\item Elmätövningarna 
      \\ -- 3 uppställningar i 3 lådor
\item Thailandsexperimenten 
(''Electrical Blackbox: Capacitive Displacement Sensor'' och
 ''Mechanical Blackbox: a cylinder with a ball inside'') 
      \\ -- 4 uppställningar av vardera experiment i 1 låda och 1 kartong
\item Estlandsexperimenten 
(''The magnetic permeability of water'' och
 ''Nonlinear Black Box'')
      \\ -- 2 uppställningar av vardera experiment i 2 lådor och 1 kartong
\item 2 diverselådor innehållande verktyg, reservbatterier och förlängningssladdar. 
\end{itemize}
Med reservation för eventuella fel i antalet lådor.

\subsubsection*{Defekt utrustning}
En väggadapter till en tongenerator i elmätövningarna fungerar
inte. Detta går att kringgå, men vi bör skaffa en ny inför nästa
läger. 

\subsubsection*{Förbrukingsartiklar}
En säkring (200\,mA) byttes i en av multimetrarna.

Det finns fortfarande många säkringar och reservbatterier kvar. Inga
förbrukningsvaror behöver köpas in. 


\section*{Genomförande}
Experimentträningslägret pågick från omkring klockan 9 på lördagen
till 17 på söndagen. Nedan visas en grov översik av planeringen.
\begin{center}
\begin{tabular}{|l|c|c|}\hline
 & förmiddag &  eftermiddag
\\\hline
lördag & elmätövningar & Thailandsexperiment
\\\hline
söndag & statistik- och mätfelsgenomgång & Estlandsexperiment 
\\\hline
\end{tabular}
\end{center}

Deltagarna hade väldigt fria händer under lägret, förut på vilka
experiment de skulle göra. När de väl har fått ett experiment och
uppgiftspapper blir de lämnade för sig själva om de inte har några
frågor. 

\subsubsection*{Elmätövningarna}
Elmätövningarna gjordes i tre grupper om två och två. Deltagarna
jobbade med elmätövningarna från klockan 10 till 13, med fikapauser. 
Det var lite för lite tid för att hinna med hela häftet. Alla
grupperna hann testa de olika momenten, men ingen av dem hann med
den sista övningen, i linjärisering.  

I min mening
är det dock inte någon större risk att linjäriseringsuppgiften inte
hanns med. Man behöver inte heller ta bort den eftersom utrustningen
redan finns och det kan vara bra med en extraövning om någon är
lite snabbare. 

\subsubsection*{Thailandsexperimenten}
Efter lunchen fick deltagarna sitta ensamma med varsitt
Thailandsexperiment. De var uppdelade så att tre jobbade med
kondensatorn och de andra tre jobbade med pendeln. 

Efter omkring 1,5--2~timmar fick de som jobbade med samma experiment
gå samman och jobba tillsammans. Efter ytterligare 20--40~minuter fick
de lösningar och rättningsmallar utdelade. Därefter bytte de experiment
och gjorde om samma procedur med det nya experimentet. 

Allt som allt pågick eftermiddagens övningar mellan klockan
14--19, även här med fikapauser. Inga lyckes helt lösa några
experiment, men det var väntat då de här övningarna mest är till för
att de ska få känna på hur IPhO kommer att bli. Det var ungefär lagom
med tid -- mycket mer än 5~timmar kan bli för mycket för att hålla
koncentrationen uppe. 


\subsubsection*{Statistik- och mätfelsgenomgång}
Jag höll en genomgång av statistik och hur man uppskattar mätfel
samt hur man propagerar osäkerheter från flera olika mätvärden till
ett slutvärde. Detta tog drygt 2~timmar till klockan 12. Genomgången
var till stor del byggd på häftet\footnotemark{} som jag skrev förra
året, och deltagarna har fått varsin kopia. 
\footnotetext{\url{
https://github.com/andsunds/Fysikpris/blob/master/experimentguide/experimentguide.pdf}}

Utav genomgången kändes det som om statistiken var på för hög nivå
för deltagarna. Till nästa år bör nog statistikdelen kortas ner 
från genomgången. Samtidigt är feluppskattningar från en mätserie svår
att presentera utan att antingen lägga en grund inom statistik eller
bara ge formlerna utan något sammanhang.

Propagering av mätfel var lättare för deltagarna att förstå rent
intuitivt, men sen är frågan om de kommer ihåg hur man gör det sen. 

Jag glömde att nämna hur IPhO behandlar upplösningen på ett
instrument. 



\subsubsection*{Estlandsexperimenten}
Här delades deltagarna upp så att två deltagare jobbade med varsin
uppställing av ''magnetic permeability of water'' (den med lasern),
och de andra fyra jobbade två och två med ''Nonlinear Black Box''.

Här jobbade de med samma experiment hela eftermiddagen, från
14--17. De var alla lite trötta på att jobba med uppgifter om svarta
lådor, så fler hade viljat jobba med lasern. Här är vi dock begränsade
av att vi bara har två uppställningar av vardera Estlandsexperiment. 

Förra året hade vi samma uppdelning av Estlandsexperimenten där varje
enskild deltagare bara fick testa på ett av experimenten. Då fick
två av deltagarna låna hem varsin ''Nonlinear Black Box''. Så gjorde
vi inte i år. 



\section*{Anmärkningar}
Över lag gick allt bra. Två punkter att tänka på till nästa år är dock
att korta ner teorigenomgången något, och att i förväg bestämma var vi
äter lunch. Vi tappade en del tid med att vela med var vi skulle äta. 

Ovanligt för i år var att vi hade sex deltagar på lägret. Detta
eftersom även två reserver var inbjudna, men bara en kom. Det är
kanske bra att bjuda in reserverna i en situation som här, där
ordinarie lagmedlemmar inte var säkra på om de skulle åka. Men om det
kommer mer än 5--7 deltagare kan det bli brist på
experimentuppställningar. 





%%%%%%%%%%%%%%%%%%%%%%%%%%%%%%%%%%%%%%%%%%%%%%%%%%%%%%%%%%%%%%%%%%%%%%
\end{document}%% ^ ^ ^ ^ ^ ^ ^ ^ ^ ^ ^ ^ ^ ^ ^ ^ ^ ^ ^ ^ ^ ^ ^ ^ ^ ^ ^
%%%%%%%%%%%%%%%%%%%%%%%%%%%%%%%%%%%%%%%%%%%%%%%%%%%%%%%%%%%%%%%%%%%%%%




%%%%  Some (useful) templates


%% På svenska ska citattecknet vara samma i både början och slut.
%% Använd två apostrofer: ''.


%% Including PDF-documents
\includepdf[pages={1-}]{filnamn.pdf} % NO blank spaces in the file name

%% Figures (pdf, png, jpg, ...)
\begin{figure}\centering
\centerline{ % centers figures larges than 1\textwidth
\includegraphics[width=.8\textwidth]{file_name.pdf}
}
\caption{}
\label{fig:}
\end{figure}

%% Figures from xfig's "Combined PDF/LaTeX"
\begin{figure}\centering
\resizebox{.8\textwidth}{!}{\input{file_name.pdf_t}}
\caption{}
\label{fig:}
\end{figure}


%% If you want to add something to the ToC
%% (Without having an actual header in the text.)
\stepcounter{section} %For example a 'section'
\addcontentsline{toc}{section}{\Alph{section}\hspace{8 pt}Labblogg} 

